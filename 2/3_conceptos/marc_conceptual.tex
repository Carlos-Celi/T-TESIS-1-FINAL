A continuación, se presentan los principales conceptos a utilizar en el presente trabajo de 
investigación.

\begin{itemize}
    \item \textbf{Inteligencia Artificial (IA)} \\
    La inteligencia artificial es una rama de las ciencias computacionales que busca emular la inteligencia humana a través de algoritmos; máquinas y sistemas realizan actividades de manera eficiente, ayudando a resolver problemas complejos en diversos campos, incluido el de la salud. \parencite{russell2004}

    \item \textbf{Machine Learning (Aprendizaje Automático)} \\
    El aprendizaje automático es una subrama de la inteligencia artificial que utiliza estadísticas e ingeniería para permitir que las máquinas mejoren su rendimiento en tareas específicas mediante el aprendizaje de datos. Se utiliza en aplicaciones de salud para analizar datos médicos y mejorar diagnósticos. \parencite{harrington2012}

    \item \textbf{Aprendizaje Supervisado} \\
    Este método de aprendizaje implica la identificación de patrones en los datos basados en un conjunto de entrenamiento etiquetado, donde los resultados esperados están presentes. Es fundamental en el desarrollo de modelos de predicción en aplicaciones de salud. \parencite{harrington2012}
    
    \item \textbf{Aplicaciones Móviles} \\
    Las aplicaciones móviles son programas diseñados para ejecutarse en dispositivos móviles como teléfonos inteligentes y tabletas. Estas aplicaciones pueden clasificarse en tres tipos principales: nativas, híbridas y web, cada una con sus propias ventajas en términos de rendimiento y compatibilidad. \parencite{herazo2023}

    \item \textbf{Desarrollo de Aplicaciones Móviles} \\
    Es el proceso de creación de software diseñado para ejecutarse en dispositivos móviles como teléfonos inteligentes y tabletas, pasando por varias etapas como diseño, desarrollo y pruebas. \parencite{mathew2023}

    \item \textbf{Aplicaciones Móviles en Salud} \\
    Las aplicaciones móviles en el sector salud permiten a los pacientes y profesionales acceder a información y servicios de manera ágil, facilitando el monitoreo y la gestión de la salud. \parencite{reanima2023}

    \item \textbf{Android Studio} \\
    Es el entorno de desarrollo integrado (IDE) oficial de Google para la creación de aplicaciones en Android, lanzado en 2013. Android Studio ofrece herramientas avanzadas como un emulador de dispositivos y un editor visual para optimizar el proceso de desarrollo. \parencite{developerandroid2023, santaella2023}

    \item \textbf{Swift} \\
    Swift es un lenguaje de programación creado por Apple en 2014 para el desarrollo de aplicaciones en sus plataformas como iOS y macOS. Este lenguaje se caracteriza por su seguridad y rendimiento optimizado, ideal para el desarrollo de aplicaciones móviles en salud. \parencite{apple2023, fernandez2023}

    \item \textbf{Cáncer} \\
    El cáncer es un conjunto de enfermedades caracterizadas por el crecimiento descontrolado de células anormales. Puede desarrollarse en cualquier parte del cuerpo y propagarse a través del sistema linfático o sanguíneo. \parencite{institutonacionalcancer2023}

    \item \textbf{Cáncer de Mama} \\
    Tipo de cáncer que afecta principalmente a las mujeres y es uno de los más comunes. Se desarrolla cuando las células de la mama crecen de manera descontrolada, formando un tumor. Los síntomas incluyen bultos en el seno, cambios en la forma y dolor. \parencite{mayoclinic2023}

    \item \textbf{Psicología en Pacientes Oncológicos} \\
    La psicooncología es una especialidad que se enfoca en el apoyo psicológico para pacientes con cáncer, ayudándoles a manejar el estrés, la ansiedad y la depresión asociados al diagnóstico y tratamiento. \parencite{oncosalud2023, areahumana2023, terapify2023}

    \item \textbf{Base de Datos} \\
    Sistema que permite almacenar y gestionar información de manera estructurada, facilitando su recuperación y análisis. Las bases de datos médicas son esenciales en aplicaciones de salud para manejar grandes volúmenes de información clínica. \parencite{perez2007, nutanix2023}



\end{itemize}
