\subsection{Aplicaciones móviles }
Las aplicaciones móviles son programas diseñados específicamente para ejecutarse en dispositivos móviles como teléfonos inteligentes y tabletas. Estas aplicaciones, que pueden variar en funcionalidad, incluyen desde entretenimiento hasta educación y salud, y permiten a los usuarios acceder a información y servicios de manera rápida y eficiente. Según \parencite{herazo2023}, una aplicación móvil es un tipo de software que se descarga y utiliza en un dispositivo móvil, que puede ser un smartphone o una tablet, y que facilita la realización de tareas específicas o la interacción con servicios y contenidos en línea en varios. Se pueden clasificar en tres categorías principales:

Se pueden clasificar en tres categorías principales:

\begin{itemize}
    \item \textbf{Aplicaciones nativas}: Están desarrolladas para un sistema operativo específico, como Android o iOS. Al estar diseñadas para una plataforma específica, las aplicaciones nativas ofrecen un rendimiento óptimo y una integración completa con el hardware y las funcionalidades del dispositivo.
    
    \item \textbf{Aplicaciones web}: Estas aplicaciones son accesibles a través de navegadores web y se ejecutan directamente en el navegador del dispositivo, lo que permite que sean multiplataforma. Aunque no requieren instalación y pueden funcionar en cualquier dispositivo, su rendimiento y funcionalidad pueden estar limitados en comparación con las aplicaciones nativas.
    
    \item \textbf{Aplicaciones híbridas}: Son una combinación de aplicaciones nativas y web, desarrolladas con tecnologías como React Native o Flutter, y permiten compartir gran parte de su código entre plataformas. Las aplicaciones híbridas ofrecen una experiencia similar a las aplicaciones nativas, pero con un menor costo de desarrollo y mantenimiento al utilizar una única base de código.
\end{itemize}

 \subsection{Inteligencia artificial}

 La inteligencia artificial (IA) es una disciplina de la informática dedicada a desarrollar sistemas capaces de realizar tareas que, en condiciones normales, requerirían inteligencia humana. Entre estas tareas se incluyen el reconocimiento de patrones, la toma de decisiones y el aprendizaje. \parencite{russell2004} definen la IA como el estudio de los agentes que perciben su entorno y toman acciones que maximizan sus posibilidades de éxito en alguna tarea.
En el ámbito de la salud, la IA se aplica en el análisis de grandes volúmenes de datos clínicos, lo que permite a los sistemas identificar patrones que facilitan el diagnóstico y la personalización de tratamientos. Según Google, la IA generativa, una subrama de la IA, permite a los modelos crear contenido nuevo a partir de datos existentes, lo cual es especialmente útil para generar imágenes médicas sintéticas y para la creación de asistentes virtuales que interactúan con los pacientes de forma personalizada \parencite{googlecloud2023}.

\subsubsection{Inteligencia artificial generativa}:

La IA generativa se enfoca en la creación de nuevo contenido a partir de patrones aprendidos en datos previos. Esta tecnología es capaz de generar textos, imágenes y otros tipos de contenido que imitan los patrones de los datos originales, lo cual la hace valiosa en aplicaciones médicas, especialmente en la creación de datos sintéticos para entrenar otros sistemas de IA. La plataforma de Google Cloud destaca que esta tecnología puede asistir en "la creación de asistentes virtuales en aplicaciones de salud" para apoyar a los pacientes y brindar soporte emocional en tiempo real \parencite{googlecloud2023} .

\subsection{Aprendizaje automático}

El aprendizaje automático (machine learning) es una técnica de IA que permite a los sistemas aprender de los datos sin una programación explícita para cada tarea específica. En lugar de ser programados con instrucciones precisas, estos sistemas emplean algoritmos que procesan datos y mejoran sus predicciones o clasificaciones a medida que aprenden de ejemplos previos. Según Oracle, el aprendizaje automático se basa en tres enfoques principales: el aprendizaje supervisado, no supervisado y por refuerzo, y cada uno tiene aplicaciones específicas en el diagnóstico y análisis de datos médicos \parencite{oracle2023}.

\subsection{Procesamiento del lenguaje natural}

El procesamiento del lenguaje natural (PLN) es una rama de la IA que permite a los sistemas entender y generar lenguaje humano de manera comprensible. En el contexto de la salud, el PLN es utilizado para desarrollar chatbots y asistentes virtuales que pueden interactuar con los pacientes de forma empática y personalizada, respondiendo a sus preguntas y brindando apoyo emocional. IBM define el PLN como "la tecnología que permite a las computadoras procesar y analizar grandes cantidades de lenguaje humano" y señala que esta capacidad es esencial para mejorar la experiencia del paciente en aplicaciones de soporte emocional y salud \parencite{ibm2023} .

\subsection{Base de datos}:
Las bases de datos son sistemas que permiten almacenar, gestionar y recuperar grandes volúmenes de información de manera eficiente, poseen una estructura definida y representan entidades y las relaciones que existen entre ellas. Estas bases de datos o repositorios contienen información ordenada que puede ser consultada a través de diversos usuarios \parencite{perez2007}. Una base de datos puede definirse como "un sistema organizado de almacenamiento de datos que permite la administración, la seguridad y el acceso rápido a los datos" \parencite{nutanix2023}.

Las bases de datos se caracterizan por ofrecer:
\begin{itemize}
    \item \textbf{Organización estructurada}: Los datos se organizan en tablas compuestas por columnas y filas, lo que permite una categorización y recuperación de la información de manera rápida y eficiente.
    \item \textbf{Reducción de redundancia}: Las bases de datos están diseñadas para minimizar la duplicidad de datos, asegurando que cada pieza de información se almacene solo una vez.
    \item \textbf{Consultas complejas}: Permiten realizar búsquedas avanzadas, consultas y análisis de datos a través de lenguajes de consulta como SQL (Structured Query Language), lo que facilita la obtención de información específica dentro de grandes volúmenes de datos.
    \item \textbf{Seguridad y control de acceso}: Implementan medidas de seguridad que controlan el acceso a los datos, permitiendo que solo usuarios autorizados puedan visualizar o modificar la información.
    \item \textbf{Recuperación y respaldo}: Las bases de datos suelen contar con mecanismos de respaldo y recuperación, lo que garantiza que la información esté protegida contra posibles fallos.
\end{itemize}

Para construir una base de datos, se diseña un modelo de datos en el cual se identifican las entidades clave, las relaciones entre ellas y los atributos de cada entidad. Este proceso permite definir la estructura de la base de datos, optimizando la organización y el acceso a la información. Existen varios tipos de relaciones en una base de datos:
\begin{itemize}
    \item \textbf{Relación de uno a uno}: Cada registro en la tabla A se asocia con un único registro en la tabla B. Por ejemplo, una persona y su número de pasaporte.
    \item \textbf{Relación de uno a muchos}: Un registro en la tabla A puede estar vinculado a múltiples registros en la tabla B. Un ejemplo sería un cliente que tiene varios pedidos asociados.
    \item \textbf{Relación de muchos a muchos}: Los registros en la tabla A pueden estar vinculados con múltiples registros en la tabla B y viceversa, como estudiantes que pueden estar inscritos en varios cursos, y cursos que pueden tener varios estudiantes inscritos.
\end{itemize}

El diseño adecuado de estas relaciones es esencial para asegurar la eficiencia y escalabilidad de la base de datos, especialmente en aplicaciones de gran tamaño y con requisitos de rendimiento elevados \parencite{nutanix2023}.

\subsubsection{Base de datos no relacionadas}

Las bases de datos no relacionales, también conocidas como bases de datos NoSQL, se diferencian de las bases de datos relacionales en que no emplean un esquema de tablas tradicional y no necesitan seguir una estructura rígida de filas y columnas. Este tipo de bases de datos es particularmente útil para el almacenamiento de datos no estructurados o semiestructurados, como archivos multimedia, documentos y datos de sensores, donde la flexibilidad es esencial. Según el blog de FP Superior UFV, las bases de datos no relacionales "permiten almacenar datos de una forma más flexible y escalable, siendo ideales para manejar grandes volúmenes de información en tiempo real" \parencite{fpsuperiorufv2023}.


Existen varios tipos de bases de datos no relacionales, cada uno con una estructura específica adaptada a diferentes casos de uso:
\begin{itemize}
    \item \textbf{Bases de datos de documentos}: Organizan los datos en formato de documentos, como JSON o XML, donde cada documento contiene una estructura única y flexible que se adapta a la información específica de cada registro.
    \item \textbf{Bases de datos de clave-valor}: Almacenan datos en pares de clave y valor, permitiendo una búsqueda rápida basada en una clave única. Este tipo de base es adecuado para aplicaciones que requieren almacenamiento de datos sencillo y de rápido acceso.
    \item \textbf{Bases de datos de grafos}: Emplean nodos y aristas para representar relaciones entre entidades, siendo ideales para analizar redes sociales o sistemas de recomendaciones.
    \item \textbf{Bases de datos de columnas}: Organizan los datos en columnas en lugar de filas, lo cual mejora el rendimiento en consultas analíticas de gran escala.
\end{itemize}

En aplicaciones de salud, las bases de datos no relacionales ofrecen una gran flexibilidad y escalabilidad, permitiendo manejar datos complejos y variados, como registros de pacientes, datos de dispositivos de monitoreo en tiempo real y grandes volúmenes de información generados por sistemas de inteligencia artificial. Amazon Web Services (AWS) destaca que las bases de datos NoSQL permiten manejar "grandes cantidades de datos y realizar consultas rápidas y escalables", lo que es esencial para aplicaciones que requieren respuestas en tiempo real \parencite{aws2023}.

Microsoft Azure, en su guía de arquitectura de datos, también señala que las bases de datos no relacionales son particularmente efectivas en entornos de big data, donde los datos se generan rápidamente y en formatos variados. Este tipo de bases de datos es adecuado para escenarios donde el esquema puede evolucionar con el tiempo y no es necesario seguir una estructura fija como en las bases de datos relacionales \parencite{microsoft2023}.

La flexibilidad de las bases de datos no relacionales las convierte en una opción ideal para aplicaciones móviles de salud, donde se requiere un almacenamiento rápido y dinámico que permita adaptarse a las necesidades cambiantes de los usuarios y los datos médicos que se recogen continuamente.

\subsection{Cáncer}

El cáncer es un conjunto de enfermedades caracterizadas por el crecimiento descontrolado y la propagación de células anormales en el organismo. Según el Instituto Nacional del Cáncer, el cáncer "puede comenzar en casi cualquier parte del cuerpo y es causado por cambios en los genes que regulan el crecimiento y la división celular" \parencite{institutonacionalcancer2023}. Las células cancerosas tienen la capacidad de invadir tejidos adyacentes y pueden diseminarse a otras partes del cuerpo a través del sistema linfático o el torrente sanguíneo en un proceso conocido como metástasis.

La Organización Mundial de la Salud (OMS) señala que el cáncer es una de las principales causas de muerte en el mundo, con aproximadamente 10 millones de muertes al año. Entre los factores de riesgo de desarrollar cáncer se incluyen el consumo de tabaco, la falta de actividad física, la dieta poco saludable y el consumo excesivo de alcohol. La OMS enfatiza la importancia de la detección temprana y del tratamiento oportuno para mejorar las tasas de supervivencia de los pacientes \parencite{oms2023}.

\subsubsection{Cáncer de Mama}

El cáncer de mama es uno de los tipos de cáncer más comunes en mujeres y es la principal causa de muerte por cáncer en este grupo a nivel mundial. El cáncer de mama se desarrolla cuando las células de la mama comienzan a crecer de manera descontrolada, formando un tumor que puede ser benigno o maligno. Según la Mayo Clinic, "el cáncer de mama puede aparecer en hombres y mujeres, pero es mucho más frecuente en las mujeres" \parencite{mayoclinic2023}.

Los síntomas del cáncer de mama incluyen:
\begin{itemize}
    \item Un bulto o engrosamiento en el seno que se siente diferente al tejido circundante.
    \item Cambios en el tamaño, forma o apariencia de la mama.
    \item Secreción anormal del pezón, que puede ser sanguinolenta.
    \item Dolor en la mama o el pezón.
    \item Cambios en la piel del seno, como enrojecimiento o formación de hoyuelos.
\end{itemize}

La detección temprana es crucial para mejorar el pronóstico en los casos de cáncer de mama. La autoexploración, las mamografías y los exámenes clínicos de mama son herramientas esenciales en la detección temprana de esta enfermedad.

\subssection{Psicología en Pacientes Oncológicos}

El diagnóstico y tratamiento del cáncer tienen un impacto significativo en la salud emocional y mental de los pacientes. La psicología oncológica, o psicooncología, es una especialidad que aborda las necesidades emocionales de las personas que enfrentan el cáncer, proporcionando herramientas para ayudarles a gestionar el miedo, la ansiedad y la depresión que pueden surgir en el proceso. Según Oncosalud, "el psicólogo cumple un rol fundamental en el tratamiento del cáncer, ayudando al paciente a afrontar los retos emocionales y psicológicos que acompañan a la enfermedad" \parencite{oncosalud2023}.

El apoyo psicológico en pacientes con cáncer no solo beneficia al paciente, sino también a sus familiares, quienes también pueden experimentar un gran estrés emocional debido a la enfermedad de su ser querido. La intervención psicológica puede mejorar la calidad de vida del paciente, ayudándole a mantener una actitud positiva frente a los tratamientos y a reducir los síntomas de depresión y ansiedad. De acuerdo con Área Humana, "el apoyo psicológico a los pacientes oncológicos permite a estos sentirse acompañados en su proceso, lo que ayuda a disminuir la angustia y a mejorar su bienestar general" \parencite{areahumana2023}.

La psicooncología incluye diversas técnicas y terapias adaptadas a las necesidades de cada paciente. Algunas de estas incluyen la terapia cognitivo-conductual, que ayuda a los pacientes a identificar y modificar pensamientos negativos, y la terapia de aceptación y compromiso, que se centra en aceptar la realidad de la enfermedad y trabajar en objetivos de vida significativos. Terapify menciona que "el tratamiento psicológico para pacientes con cáncer, conocido como psicooncología, se adapta a cada fase de la enfermedad, proporcionando apoyo en el diagnóstico, durante el tratamiento y en la recuperación" \parencite{terapify2023}.


\subsection{Desarrollo de Aplicaciones Móviles}

El desarrollo de aplicaciones móviles es el proceso de creación de software diseñado para ejecutarse en dispositivos móviles, como teléfonos inteligentes y tabletas. Este proceso involucra varias etapas, desde la concepción de la idea hasta el diseño, desarrollo, pruebas y despliegue de la aplicación en las tiendas de aplicaciones. Según Mathew y Clark Kimberly, "el desarrollo de aplicaciones móviles es el conjunto de procesos y procedimientos que implican la escritura de software para dispositivos inalámbricos pequeños, como teléfonos inteligentes y otros dispositivos portátiles" \parencite{mathew2023}. El desarrollo de aplicaciones móviles puede dividirse en tres tipos principales: desarrollo de aplicaciones nativas, híbridas y web, cada uno de los cuales ofrece diferentes ventajas y desafíos en términos de rendimiento, compatibilidad y costos de desarrollo.

\subsubsection{Desarrollo de Aplicaciones Móviles en la Salud}

El desarrollo de aplicaciones móviles para el sector salud ha crecido significativamente en los últimos años, proporcionando herramientas innovadoras para mejorar el acceso a la atención médica y el monitoreo de la salud de los pacientes. Estas aplicaciones permiten a los profesionales de la salud y a los pacientes acceder a información y servicios de manera rápida y conveniente, lo cual ha transformado el sistema de atención sanitaria. De acuerdo con Reanima Soluciones, las aplicaciones móviles de salud "brindan acceso a servicios médicos, permiten el monitoreo de pacientes y facilitan la recopilación y análisis de datos médicos en tiempo real" \parencite{reanima2023}.

Las aplicaciones de salud ofrecen funcionalidades como la consulta en línea, la gestión de citas, el monitoreo de signos vitales y el acceso a historiales médicos, lo cual permite una atención más eficiente y personalizada. Para desarrollar una aplicación de salud efectiva, es crucial seguir ciertos pasos, como definir los objetivos de la app, seleccionar las herramientas adecuadas para el desarrollo y cumplir con las normativas de seguridad y privacidad de datos. Según GooApps, una guía completa para el desarrollo de una aplicación de salud debe incluir "la definición de funcionalidades, la selección de tecnologías apropiadas y la validación de la app mediante pruebas exhaustivas para asegurar su usabilidad y seguridad" \parencite{gooapps2022}.

El desarrollo de aplicaciones móviles en el ámbito de la salud no solo ha mejorado la accesibilidad a la atención médica, sino que también ha permitido un seguimiento continuo y en tiempo real de los pacientes, especialmente aquellos con enfermedades crónicas. Este tipo de aplicaciones ha demostrado ser eficaz en la gestión de enfermedades, el seguimiento de tratamientos y la promoción de hábitos saludables, permitiendo una atención más cercana y un control personalizado de la salud del paciente.

\subsection{Herramientas de Desarrollo de Aplicaciones Móviles: Android Studio y Swift}

El desarrollo de aplicaciones móviles requiere herramientas especializadas que permitan crear aplicaciones eficientes, seguras y adaptadas a diferentes plataformas. Android Studio y Swift son dos de las herramientas más populares y utilizadas para el desarrollo de aplicaciones móviles, especialmente en sistemas operativos como Android e iOS, respectivamente.

\subsubsection{Android Studio}

Android Studio es el entorno de desarrollo integrado (IDE) oficial de Google para el desarrollo de aplicaciones en el sistema operativo Android. Lanzado en 2013, Android Studio proporciona una serie de herramientas que permiten a los desarrolladores crear, probar y depurar aplicaciones de manera eficiente. Entre sus características principales, Android Studio incluye un editor de código avanzado, un emulador de dispositivos, y un sistema de gestión de versiones que facilita el control de cambios en el desarrollo del software. Según Developer Android, Android Studio está diseñado para "ofrecer las herramientas más rápidas para la creación de aplicaciones en todos los tipos de dispositivos Android" \parencite{developerandroid2023}.

Android Studio está basado en IntelliJ IDEA y permite programar principalmente en Java y Kotlin, dos lenguajes compatibles con Android. Jesús Santaella (2023) destaca que Android Studio "cuenta con características avanzadas como un editor visual para el diseño de interfaces de usuario, un emulador rápido y una herramienta de análisis de rendimiento", lo cual permite optimizar la experiencia del usuario en aplicaciones móviles \parencite{santaella2023}.

\subsubsection{Swift}

Swift es un lenguaje de programación desarrollado por Apple en 2014 para la creación de aplicaciones en sistemas operativos como iOS, macOS, watchOS y tvOS. Este lenguaje es moderno, seguro y rápido, diseñado para ser fácil de aprender y de usar. Swift permite a los desarrolladores escribir código de manera eficiente y con un rendimiento optimizado, lo cual es crucial para el desarrollo de aplicaciones móviles. Apple describe Swift como "un lenguaje de programación potente e intuitivo para macOS, iOS, watchOS y tvOS, que permite escribir código más seguro y mejorar el rendimiento de las aplicaciones" \parencite{apple2023}.

Swift está diseñado para ser intuitivo y flexible, y es compatible con el marco de trabajo Xcode, el IDE oficial de Apple para el desarrollo de aplicaciones. Según Esteban Fernández (2023), Swift se caracteriza por su sintaxis sencilla y su enfoque en la seguridad, lo cual "reduce errores comunes y mejora la estabilidad de las aplicaciones, especialmente en plataformas móviles" \parencite{fernandez2023}.

Ambas herramientas, Android Studio y Swift, son esenciales para el desarrollo de aplicaciones móviles en sus respectivos sistemas operativos y permiten a los desarrolladores crear aplicaciones optimizadas y personalizadas para mejorar la experiencia del usuario en dispositivos móviles.
